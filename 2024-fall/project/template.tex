\documentclass{article}
% ready for submission
\usepackage[preprint]{template}
\usepackage[utf8]{inputenc} % allow utf-8 input
\usepackage[T1]{fontenc}    % use 8-bit T1 fonts
\usepackage{hyperref}       % hyperlinks
\usepackage{url}            % simple URL typesetting
\usepackage{booktabs}       % professional-quality tables
\usepackage{amsfonts}       % blackboard math symbols
\usepackage{nicefrac}       % compact symbols for 1/2, etc.
\usepackage{microtype}      % microtypography
\usepackage{xcolor}         % colors

\title{Instructions for CS-GA 2565 Final Course Project}

\author{
  Student Name 1 \\
  Affiliation \\
  Address \\
  \texttt{email} \\
  \And
  Student Name 2 \\
  Affiliation \\
  Address \\
  \texttt{email} \\
  \And
  Student Name 3 \\
  Affiliation \\
  Address \\
  \texttt{email} \\
  \AND
  Team [ID] \\
}

\begin{document}

\maketitle

\begin{abstract}
  This document outlines the instructions for the final course project of CS-GA 2565 Fall 2024. Your project proposal and final report should use this space for abstract.
\end{abstract}

\section{Final Course Project}
The final course project constitutes 30\% of your overall grade. The objective is to apply the machine learning concepts acquired during this course to a real-world problem. Choose a pertinent and applicable issue, identify an appropriate data source for your machine learning solution, and if no suitable data source exists, propose methods to gather the required data efficiently. You need to think about why machine learning can help solve your problem and whether you have enough high-quality data. You need to survey literature on existing solutions on this types of problem, and identify the machine learning challenges that make the problem hard and interesting. Towards the challenges you identified, what additional techniques or algorithmic improvements would you apply to your solution? Do these proposed technical components really help in comparison to simpler baselines? Make sure you demonstrate the findings through careful experimentation, hyperparameter searches, and present the results using clear tables and figures.

\section{Key Dates and Details}
Please mark your calendars with these key dates:
\begin{itemize}
  \item \textbf{Oct 15, 2024}: Form groups of three students. Sign-up link is \url{https://forms.gle/nNEHn1k1dQv5rJ3W7}. Sign-up is due at 11:59PM. Groups will be assigned a number. Students not part of a group will be assigned a group by the instructor.
  \item \textbf{Oct 31, 2024}: Submit project proposals on Gradescope by 12PM. While this proposal is mandatory, it will not be graded. Its intent is to establish a checkpoint. With the submission, book a mandatory consultation with the instructor to discuss/approve your selected topic and proposed methodologies. This consultation can be outside of regular office hours. Calendar booking link: \url{https://calendar.app.google/iNPzAVEHR6tnTSWf9} (Link will be activated between Oct 16 - Nov 15).
  \item \textbf{Nov 15, 2024}: Complete at least one consultation with the instructor.
  \item \textbf{Dec 9, 2024}: Submit the project presentation slides by 11:59PM. The submission link is \url{https://forms.gle/Dw8b9JCeJuBxCy8r9}.
  \item \textbf{Dec 10, 2024}: Course project presentation from 4:55PM to 6:55PM.
  \item \textbf{Dec 13, 2024}: Final report submission on Gradescope by 12PM.
  \item \textbf{Dec 14, 2024}: Complete the self and peer evaluation via Google Form. This is crucial for determining your contribution to the project. Evaluation link: \url{https://forms.gle/geMc7ZKxmyostPQe9}.
\end{itemize}

\section{Submission Guidelines}
\paragraph{Page format.} Adhere to the LaTeX template provided with \textbf{this document} (\texttt{template.sty}). Avoid altering font type, size, line spacing, margins, or heading arrangements. Use standard LaTeX structures. For guidance on LaTeX, refer to \url{https://www.latex-project.org/get/} or utilize the online editor Overleaf at \url{https://www.overleaf.com/}. \textcolor{red}{Note: Not using the provided template (including fonts, headings, margins, etc.) will result in an automatic fail on the project.}

\paragraph{Citation format.} Employ BibTeX for citations. An illustrative reference is \cite{vaswani2017attention}. The references for this template reside in \texttt{template.bib}.

For the project proposal:
\begin{itemize}
  \item \textbf{Page limit:} 2 pages, excluding references. \textcolor{red}{Note: Exceeding the page limit is not acceptable. Please stay under the page limit.}
  \item \textbf{References:} No page limit.
\end{itemize}

For the final project report:
\begin{itemize}
  \item \textbf{Page limit:} 8 pages, excluding references, including tables and figures. \textcolor{red}{Note: Exceeding the page limit is not acceptable. Please stay under the page limit. It is ok to have less than 8 pages, but if the report is significantly below the limit, then it may also be graded unfavorably due to a lack of content.}
  \item \textbf{References:} No page limit.
  \item \textbf{Code:} Submit your code as a zip file.
\end{itemize}

\section{Report Structure}
Your proposal should contain the following sections:
\begin{itemize}
  \item \textbf{Abstract}
  \item \textbf{Introduction}
  \item \textbf{Related Work}
  \item \textbf{Proposed Dataset and Approach}
  \item \textbf{Expected Results and Milestones}
\end{itemize}

The final report should contain the following sections:
\begin{itemize}
  \item \textbf{Abstract}
  \item \textbf{Introduction}
  \item \textbf{Related Work}
  \item \textbf{Method}
  \item \textbf{Dataset}
  \item \textbf{Experiments}
  \item \textbf{Discussion/Conclusion}
\end{itemize}

\section{Presentation}
Presentations will be held during the course's final lecture. Prepare a slide deck and submit the slides as a PDF a day prior to the presentation. Aim for a duration of 5 minutes (around 5 slides), followed by a 1-minute Q\&A session. The precise time allocation may vary based on the total number of groups. The exact time allocation will be announced.

\section{Grading}
Your project will be graded out of 30 points, distributed in Table~\ref{tab:grading}.

\begin{table}[h!]
\caption{Grading guideline}
\label{tab:grading}
\begin{tabular}{llp{3.5in}}
\toprule
\textbf{Area} & \textbf{Points} & \textbf{Criteria} \\
\hline
Topic Selection & 3 & Relevance to ML, applicability, and novelty. \\
Literature Review & 3 & Comprehensive survey, proper citation, and insightful connections. \\
Dataset Selection & 5 & Suitability for ML, quality, preprocessing efforts, and data collection processes. \\
Modeling & 5 & Appropriateness of the proposed method, clarity of the learning objective and optimization methods, and mathematical correctness. \\
Experiments & 5 & Thoroughness, clarity of results, and effective analyses. \\
Writing & 3 & Clarity, structure, and motivation. \\
Presentation & 4 & Clarity, organization, time management, and content depth. \\
Participation & 2 & Attendance during presentations and participation during Q\&A.\\
\bottomrule
\end{tabular}
\end{table}

\bibliographystyle{unsrt}
\bibliography{template}

\end{document}